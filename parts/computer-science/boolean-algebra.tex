\chapter{Boolean algebra}
\section{Boolean expressions}
A way to convert a truth table into a Boolean expression is by writing an
expression that evaluates to true for each and every row that evaluates to true
and chain them together with the OR operator. This expression may later be
simplified through logic identities.

Any Boolean function can be expressed by an expression containing only AND and
NOT operations. For instance, \(A \lor B\) can be expressed as \(\lnot \left(
\left( \lnot A \right) \land \left( \lnot B \right) \right)\).

\subsection{de Morgan's Laws}
De Morgan's laws are a pair of transformation rules that are both valid rules of
inference. The rules allow the expression of conjunctions and disjunctions
purely in terms of each other via negation. In mathematical notation, the rules
state that
\[\lnot \left( A \land B \right) =
  \left( \lnot A \right) \lor \left( \lnot B \right)\]
and
\[\lnot \left( A \lor B \right) =
  \left( \lnot A \right) \land \left( \lnot B \right)\]
are valid equalities.
