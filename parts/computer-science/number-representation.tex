\chapter{Number representation}

\section{Fixed point}
Fixed point is a form of representing real numbers with a fixed number of digits
after the radix point.

\subsection{Fixed point conversion exercises}
\paragraph{Problem.}
Convert \texttt{01101011} and \texttt{11001100} from fixed
point notation with four bits for the integral part and four bits for the
fractional part to decimal first using unsigned arithmetic and then using two's
complement.
\paragraph{Solution.}
Using unsigned arithmetic,
\[0110.1011_{2} = 4_{10} + 2_{10} + 0.5_{10} + 0.125_{10} + 0.0625_{10}
  = 6.6875_{10}\]
\[1100.1100_{2} = 8_{10} + 4_{10} + 0.5_{10} + 0.25_{10} = 12.75_{10}\]

Using two's complement,
\[0110.1011_{2} = 4_{10} + 2_{10} + 0.5_{10} + 0.125_{10} + 0.0625_{10}
 = 6.6875_{10}\]
\[1100.1100_{2} = \left( 8_{10} + 4_{10} \right) - 16_{10} + 0.5_{10}
 + 0.25_{10} = -3.25_{10}\]

\paragraph{Problem.}
Evaluate the following expressions using 8-bit unsigned
fixed point arithmetic. The results should have 4 bits for the integral part and
4 bits for the fractional part. The results must be rounded if necessary.
Overflow and rounding must be indicated.

\begin{enumerate}
    \item \(0111.0110_{2} + 0010.0010_{2}\)
    \item \(1000.1001_{2} - 1001.1000_{2}\)
    \item \(11.101011_{2} + 001100.00_{2}\)
\end{enumerate}

\paragraph{Solution.}
\begin{enumerate}
    \item \(0111.0110_{2} + 0010.0010_{2} = 1001.1000_{2}\)
    \item \(1000.1001_{2} - 1001.1000_{2} = 0010.0001_{2}\) \textbf{(overflow)}
    \item \(11.101011_{2} + 001100.00_{2} = 1111.1011_{2}\) \textbf{(rounding)}
\end{enumerate}

\section{Floating Point}
Floating point is a form of representing real numbers while supporting a
trade-off between range and precision.

All exact values a floating point may assume are of the form
\[\text{significand} \cdot \text{base}^\text{exponent}\]
where significand, base, and exponent are integers and base is greater than or
equal to two.

\subsection{Usage}
Albeit more computationally expensive than fixed point arithmetic, floating
point arithmetic is more commonly used in modern times. The main reasons for
this is that the same floating point format is suitable for representing a wider
range of values than its fixed point counterpart.

Supercomputers are often compared in terms of FLOPS. FLOPS stands for
floating-point operations per second. It is a more accurate measure of computer
performance for computers which make heavy usage of floating-poing
operations.

\subsection{IEEE 754}
The IEEE 754 is the IEEE standard for floating-point arithmetic. It is a
technical standard for floating-point computation established in 1985.

The value of a single precision IEEE-754 number is given by
\[\text{sign} \cdot \text{significand} \cdot 2^\text{exponent}\]

\subsubsection{IEEE 754 Number Formats}
The standard defines five number formats, two of which are in wide usage:
binary32 and binary64.

The binary32 format defines 32-bit floating point numbers in base 2. These are
also known as \textbf{single precision floating point numbers}. This format
delegates one bit to the sign, 8 bits to the exponent and the remaining 23 bits
to the significand.

The binary64 format defines 64-bit floating point numbers in base 2. These are
also known as \textbf{double precision floating point numbers}. This format
delegates one bit to the sign, 11 bits to the exponent and the remaining 52 bits
to the significand.

\subsubsection{IEEE 754 Conversion Algorithm}
In order to convert the decimal representation of a real number to an IEEE 754
number the following algorithm may be used.
\begin{enumerate}
\item Set the first bit if the number is negative.
\item Find the biggest power of two which is less than or equal to the absolute
    value of the number and express it in the exponent bits biased by 127.
\item Express the fractional part of the quotient between the number whose
    representation is of interest and the abovementioned power of two in the
    significand bits.
\end{enumerate}

An algorithm for the conversion from the IEEE 754 format to decimal is not
described as it is much more obvious than the one required for the conversion
from decimal values to this floating point format.

\subsubsection{IEEE 754 Conversion Exercises}
\paragraph{Problem.}
Convert -7 to a single-precision IEEE 754 floating point number.
\paragraph{Solution.}
\begin{enumerate}
\item It is negative, therefore the first bit is set.
\item The biggest power of two less than or equal to the absolute value is
    \(2^2 = 4\). Therefore, the exponent should be \(127 + 2 = 129\).
\item Determine the significand: \(7 \div 4 = 1.75\).
\item This results in \singleprecision{11000001011000000000000000000000}.
\end{enumerate}

\paragraph{Problem.}
Convert 255 to a single-precision IEEE 754 floating point number.
\paragraph{Solution.}
\begin{enumerate}
\item It is positive, therefore the first bit is unset.
\item The biggest power of two less than or equal to the absolute value is
    \(2^7 = 128\). Therefore, the exponent should be \(127 + 7 = 134\).
\item Determine the significand: \(255 \div 128 = 1.9921875\).
\item This results in \singleprecision{01000011011111110000000000000000}.
\end{enumerate}

\paragraph{Problem.}
Convert \singleprecision{01000110110000000000000000000000} from IEEE 754 single
precision floating point to a decimal number.
\paragraph{Solution.}
\begin{enumerate}
\item The first bit is unset, therefore it is positive.
\item The exponent is \(141 - 127 = 14\).
\item The significand is \(1 + 2^{-1} = 1.5\).
\item This results in \(1.5 \times 2^{14} = 1.5 \times 16384 = 24576\).
\end{enumerate}

\paragraph{Problem.}
Convert \singleprecision{11001011010100000000000011000000} from IEEE 754 single
precision floating point to a decimal number.
\paragraph{Solution.}
\begin{enumerate}
\item The first bit is set, therefore it is negative.
\item The exponent is \(150 - 127 = 23\).
\item The significand is \(1 + 2^{-1} + 2^{-3} + 2^{-16} + 2^{-17} =
    1.62502288818359375\).
\item This results in \(-1.62502288818359375 \times 2^{23} = 13,631,680\).
\end{enumerate}

\textbf{Finding Representation Limits}

\textit{Find the largest positive number that can be represented with a single
precision IEE 754 floating point number.}

Straightforwardly enough, this number has the largest possible significand and
the largest possible exponent of a normalized value.

Therefore, this number is \singleprecision{01111111011111111111111111111111},
which in decimal is

\begin{align}
    &\left(1 + \sum_{i=1}^{23} 2^{-i}\right) \times 2^{127} \nonumber \\
    &= \frac{16777215}{8388608} \times 2^{127} \nonumber \\
    &= 340282346638528859811704183484516925440 \label{largest-positive-32} \\
    &\approx 3.403 \times 10^{38} \nonumber
\end{align}

\textit{Find the second largest positive number that can be represented with a
single precision IEEE 754 floating point number.}

This number is easily derived from the largest positive number, which is
\ref{largest-positive-32}. To obtain the first number smaller than the current
number, one should subtract one from the significand bits. However, if the
significand is 1.0, one should make the significand the biggest possible value
it may assume and decrement the exponent\footnote{Take the two options as
alternatives to obtain a number smaller than the largest positive representable
number. Decrementing the exponent will produce a number equal to the product of
\(0.5\) and the largest positive number. However, decrementing the significand
bits would produce a number approximately equal to the product of
\(0.999999881\) and largest positive number. This clearly demonstrates that the
sought after number is found by decrementing the significant bits.}.

Therefore, the second largest positive number that can be represented with a
single precision IEEE 754 floating point number is

% Could have used align*, but let us keep using align in case we need labels in
% the future.
\begin{align}
    \left(1 + \sum_{i=1}^{22} 2^{-i}\right) \times 2^{127}
    &= \frac{8388607}{4194304} \times 2^{127} \nonumber \\
    &= 340282326356119256160033759537265639424 \nonumber \\
    &\approx 3.403 \times 10^{38} \nonumber
\end{align}

This also shows that the difference between the largest possible number and the
second largest possible number is approximately \(2.03 \times 10^{31}\).

\textit{Find the smallest positive number that can be represented with a single
precision IEE 754 floating point number.}

This number is the product of the smallest positive significand by the power of
the base raised to the smallest normalized exponent.

Namely, this number is \singleprecision{00000000100000000000000000000000}, whose
decimal representation is

\begin{align}
    2^{-126}
    &= \frac{1}{85070591730234615865843651857942052864}
    \label{smallest-positive-32} \\
    &\approx 1.175 \times 10^{-38} \nonumber
\end{align}

\textit{Find the second smallest positive number that can be represented with a
single precision IEE 754 floating point number.}

The opposite of what was done to find the second largest positive number may be
done to find the second smallest positive number. Therefore, the significand is
made \(1 + 2^{-23}\) and the exponent is left unchanged.

\begin{align}
    &\left(1 + 2^{-23}\right) \times 2^{-126} \nonumber \\
    &= \frac{8388609}{713623846352979940529142984724747568191373312} \nonumber
    \\ &\approx 1.175 \times 10^{-38} \nonumber
\end{align}

This also shows that the difference between the smallest possible number and the
second smallest possible number is approximately \(1.401 \times 10^{-45}\).

\textit{Find the sum of the following floating point numbers.}
\begin{itemize}
    % Remove bullets
    \item[] \singleprecision{11000011010100000000000000000000}
    \item[] \singleprecision{01000110110000000000000000000000}
\end{itemize}

As the exponent of the positive number is greater, the result is positive.
Evaluating the addition is easier if the significands are aligned. In order to
achieve this, denormalize the negative value so that its exponent is also 14.
The \textbf{actual} significand then changes in the following way:

\begin{itemize}
    % Remove bullets
    \item[] \texttt{1.1010000000}
    \item[] \texttt{0.0000001101}
\end{itemize}

Therefore, the operation on the significands is a simple binary subtraction.
\[\texttt{1.1000000000} - \texttt{0.0000001101} = \texttt{1.0111110011}\]
This results in a normalized significand. Therefore, straightforwardly enough,
the complete result is \singleprecision{01000110101111100110000000000000}.

One might convert these numbers to decimal to better understand what is
happening. After doing so, it is found that the subtraction performed was
\(24576 - 208 = 24368\).

