\chapter{Hardware}
\section{Hard disks}
Any hard disk drive (HDD) can contain only four primary partitions. This
limitation may be circumvented with an extended partition, a special type of
primary partition that allows multiple logical partitions to be created within
it. An HDD may contain only one extended partition.

An HDD records data by magnetizing a thin film of ferromagnetic material on a
disk. Sequential changes in the direction of magnetization represent binary
data bits.

A typical HDD design consists of a spindle that holds flat circular disks, also
called platters, which hold the recorded data. The platters are made from a
non-magnetic material, usually aluminium alloy, glass, or ceramic, and are
coated with a shallow layer of magnetic material typically 10–20 nm in depth,
with an outer layer of carbon for protection.

The platters in contemporary HDDs are spun at speeds varying from 4,200 rpm in
energy-efficient portable devices, to 15,000 rpm for high-performance servers.
The first HDDs spun at 1,200 rpm and, for many years, 3,600 rpm was the norm.
Nowadays, the platters in most consumer-grade HDDs spin at either 5,400 rpm or
7,200 rpm.

Information is written to and read from a platter as it rotates past devices
called read-and-write heads that operate very close (often tens of nanometers)
over the magnetic surface. The read-and-write head is used to detect and modify
the magnetization of the material immediately under it.

\paragraph{Partial response maximum likelihood.} In computer data storage,
partial response maximum likelihood (PRML) is a method for converting the weak
analog signal from the head of a magnetic disk or tape drive into a digital
signal. PRML attempts to interpret correctly even small changes in the analog
signal, whereas peak detection relies on fixed thresholds. Because PRML can
correctly decode a weaker signal it \textbf{allows higher density recording}.

For example, PRML would read the magnetic flux density pattern 70, 60, 55, 60,
70 (where 60 is the baseline signal) as binary ``101'', and the same for 45, 40,
30, 40, 45 (baseline of 40) whereas peak detector would decode everything
above, say, 50 as high, and below 50 as low, so the first pattern would read
``111'' and the second as ``000''.
