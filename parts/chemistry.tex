\part{Chemistry}

\chapter{Gases}

Ideal gas law
\[PV = nRT\]

Diffusion. Diffusion is the rate at which two gases mix.
Effusion. Effusion is the rate at which a gas escapes through a pin hole.

Graham's law of diffusion or effusion

Graham's law states that the rate of effusion or of diffusion of a gas is
inversely proportional to the square root of its molecular weight.

\[\frac{R_1}{R_2} = \sqrt{\frac{M_2}{M_1}}\]

In the same conditions of temperature and pressure, the molar mass is
proportional to the mass density. Therefore the rate of diffusion of different
gases is inversely proportional to the square root of their mass densities.

\[\frac{R_1}{R_2} = \sqrt{\frac{M_2}{M_1}}\]

\chapter{Solutions}

Solution. A solution is a homogeneous mixture of two or more substances.

Solute is a substance that is dissolved in the solution.

Solvent is the substance that dissolves the solute. Solvent is present in
greater amount.

Concentration is the ratio of solute and solvent.

Concentration can be measured using molarity, molality and mole fraction.

\[\text{Molarity (M)} = \frac{\text{moles of solute}}{\text{liters of
solution}}\]

\[\text{Molality (m)} = \frac{\text{moles of solute}}{\text{kg of solution}}\]

Mole fraction. The mole fraction of a component in solution is the number of
moles of that component divided by the total number of moles of all components
in the solution.

\[X_A = \frac{n_A}{\sum{n_x}}\]

Dilution. Diluting a solution means adding more solvent without adding more
solute.
The following formula is quite helpful in these calculations.

\[M_i V_i = M_f V_f\]

Mole. Mole is the amount of substance that contains same number of particles as
there are atoms in 12 grams of pure carbon-12.

One mole of substance is Avogadro's number.

The number of molecules in a mole is known as Avogadro's constant.

\[N_A = 6.023 \cdot 10^{23}\]

One mole of gas has a volume of 22.4 liters at Standard Temperature and
Pressure.

Relation between moles and grams.
One mole of a substance weights the molecular weight of substance in grams.

\chapter{Atomic properties}

Ionization energy. Ionization energy is qualitatively defined as the amount of
energy required to remove the most loosely bound electron of an isolated
gaseous atom to form a cation. It is always endothermic (positive).

\chapter{Acids and bases}

The Henderson-Hasselbalch equation describes the derivation of pH as a
measure of acidity (using pKa, the negative log of the acid dissociation
constant).

\[pH = pK_a + \log_{10} \frac{[\text{A}^-]}{[\text{HA}]}\]
\[pOH = pK_b + \log_{10} \frac{[\text{BH}^+]}{[\text{B}]}\]

where [A-] is the concentration of conjugate base and [HA] is the concentration
of the acid.

\section{pH of salts}
Salts from strong bases and strong acids do not hydrolyze (pH 7).

Salts from strong bases and weak acids give a pH higher than 7 when hydrolyzed.

Salts from weak bases and strong acids give a pH lower than 7 when hydrolyzed.

\chapter{Kinetics}

One of the most important topics of chemical kinetics is the \textbf{rate law}.
The rate law is an expression relating the rate of a reaction to the
concentrations of the chemical species present, which may include reactants,
products, and catalysts. Many reactions follow a simple rate law, which takes
the form
\[v = k [A]^\alpha [B]^\beta \ldots\]
For reactions with multiple steps, the rate law, rate constant, and order are
determined by experiment, and the orders are \textbf{not} generally the same as
the stoichiometric coefficients in the reaction equation.  A final important
point about rate laws is that overall rate laws for a reaction may contain
reactant, product and catalyst concentrations, but must not contain
concentrations of intermediates.

\paragraph{Law of Mass Action.} The law of mass action is a general description
of the equilibrium condition; it defines the \textit{equilibrium constant
expression}.
\[K_C = \frac{[C]^\gamma [D]^\delta}{[A]^\alpha [B]^\beta}\]

\paragraph{Notes on the equilibrium expression.} The equilibrium expression for
a reaction is the reciprocal of that for the reaction written in reverse. When
the equation for a reaction is multiplied by n, \(K\) is raised to n.  The
units for \(K\) depend on the reaction being considered, but it is
\textbf{customarily written without units}.

\paragraph{Equilibrium in terms of pressure.} The equilibrium expression for a
reaction can be written in terms of partial pressures.
\[K_P = \frac{\left(P_C\right)^\gamma
\left(P_D\right)^\delta}{\left(P_A\right)^\alpha \left(P_B\right)^\beta}\]
\[K_P = K_C (RT)^{\Delta n}\]
where

n is the difference between the number of moles of gases in the products and in
the reactants.

\paragraph{Heterogeneous Equilibrium.} The equilibrium constant does not depend
on the amounts of pure solid
or liquids present. Just gases and aqueous solutions are taken into account.

\paragraph{The Reaction Quotient (Q).} In general, all reacting chemical
systems are characterized by their Reaction Quotient, Q. It has the same
formula as K, but can be used even when the system is not at equilibrium.

\paragraph{Le Chatelier's Principle.} \textit{If a system at equilibrium is
disturbed, the system tends to shift its equilibrium position to counter the
effect of the disturbance.}

\chapter{Organic chemistry}
Organic chemistry is dominated by the functional group approach, where organic
molecules are constructed from an inert hydrocarbon skeleton onto which
functional groups are attached.

The properties and reaction chemistry of a particular functional group are
highly independent of environment. Therefore, it is only necessary to know about
the chemistry of a few functions in order to predict the chemical behavior of
thousands of real organic chemicals.

\section{Alcohols}
\subsection{Primary alcohol}
Primary alcohols have an -OH function attached to an R-CH2- group.

Primary alcohols can be oxidised to aldehydes and on to carboxylic acids.

Primary alcohols can be shown as \texttt{RCH2OH}.

\subsection{Secondary alcohol}
Secondary alcohols have an -OH function attached to a R2CH- group.

Secondary alcohols can be oxidised to ketones.

Secondary alcohols can be shown as \texttt{R2CHOH}.

\subsection{Tertiary alcohol}
Tertiary alcohols have an -OH function attached to a R3C- group.

Tertiary alcohols are resistant to oxidation with acidified potassium
dichromate(VI), K.

Tertiary alcohols can be shown as \texttt{R3COH}.

\section{Diol or polyol}
Diols and polyols are alcohols with two or more -OH functions.

Diols and polyols are very soluble in water. They are used as high temperature
polar solvents.

\section{Aldehydes}
Aldehydes have a hydrogen and an alkyl (or aromatic) group attached to a
carbonyl function.

Aldehydes are easily oxidized to carboxylic acids, and they can be reduced to
primary alcohols.

Aldehydes can be shown as \texttt{RCHO}.

\section{Ketones}
Ketones have a pair of alkyl or aromatic groups attached to a carbonyl function.

Ketones can be shown as \texttt{RCOR}.

\section{Carboxylic acids}
Carboxylic acids have an alkyl or aromatic groups attached to a hydroxy-carbonyl
function.

Carboxylic acids can be shown as \texttt{RCOOH}.

Carboxylic acids are weak Bronsted acids.

\section{Esters}
Esters have a pair of alkyl or aromatic groups attached to a carbonyl + linking
oxygen function.

Esters can be shown as \texttt{RCOOR}.

The reaction between a carboxylic acid and an alcohol produces an ester and
water.
This is an acid catalyzed equilibrium.

\section{Amides}
\subsection{Primary amides}
Primary amides have an alkyl or aromatic group attached to an amino-carbonyl
function.

Primary amides can be shown as \texttt{RCONH2}.

\subsection{Secondary amides}
Secondary amides have an alkyl or aryl group attached to the nitrogen:
\texttt{RCONHR}.

\subsection{Tertiary amides}
Tertiary amides have two alkyl or aryl group attached to the nitrogen:
\texttt{RCONR2}.

\section{Amines}
\subsection{Primary amine}
Primary amines have an alkyl or aromatic group and two hydrogens attached to a
nitrogen atom.

Primary amines are basic functions that can be protonated to the corresponding
ammonium ion.
Primary amines are also nucleophilic.

Primary amines can be shown as \texttt{RNH2}.

\subsection{Secondary amine}
Secondary amines have a pair of alkyl or aromatic groups, and a hydrogen,
attached to a nitrogen atom.

Secondary amines are basic functions that can be protonated to the corresponding
ammonium ion.
Secondary amines are also nucleophilic.

Secondary amines can be shown as \texttt{R2NH}.

\subsection{Tertiary amine}
Tertiary amines have three alkyl or aromatic groups attached to a nitrogen atom.

Tertiary amines are basic functions that can be protonated to the corresponding
ammonium ion.
Tertiary amines are also nucleophilic.

Tertiary amines can be shown as \texttt{R3N}.

\section{Acid chlorides}
Acid chlorides, or acyl chlorides, have an alkyl (or aromatic) group attached to
a carbonyl function plus a labile (easily displaced) chlorine.

Acid chlorides highly reactive entities are highly susceptible to attack by
nucleophiles.
Acid chlorides can be shown as \texttt{RCOCl}.

\section{Acid anhydrides}
Acid anhydrides are formed when water is removed from a carboxylic acid, hence
the name.

Acid anhydrides can be shown as \texttt{(RCO)2O}.

\section{Nitriles}
Nitriles (or organo cyanides) have an alkyl (or aromatic) group attached to a
carbon-triple-bond-nitrogen function.

Nitriles can be shown as \texttt{RCN}.

Note that there is a nomenclature issue with nitriles/cyanides. If a compound is
named as the nitrile then the nitrile carbon is counted and included, but when
the compound is named as the cyanide it is not.

For instance, \texttt{CH3CH2CN} is called propane nitrile or ethyl cyanide
(cyanoethane).

\section{Carboxylate ion or salt}
Carboxylate ions are the conjugate bases of carboxylic acids, i.e. the
deprotonated carboxylic acid.

When the counter ion is included, the salt is being shown.

Salts can be shown as \texttt{RCOONa}.

\section{Amino acids}
Amino acids, strictly alpha-amino acids, have carboxylic acid, amino function
and a hydrogen attached to a the same carbon atom.

There are 20 naturally occurring amino acids. All except glycine (R = H) are
chiral and only the L enantiomer is found in nature.

Amino acids can be shown as \texttt{R-CH(NH2)COOH}.

\section{Ethers}
Ethers have a pair of alkyl or aromatic groups attached to a linking oxygen
atom.

Ethers are surprisingly unreactive and are very useful as solvents for many many
(but not all) classes of reaction.

Ethers can be shown as \texttt{ROR}.

\section{Polymer}
Polymers consist of small monomer molecules that have reacted together so as to
form a large covalently bonded structure.

There are two general types of polymerisation: addition and condensation.

Linear chain polymers are generally thermoplastic, while three dimensional
network polymers are not. \textbf{Thermoplastic polymers} become pliable or
moldable above a specific temperature and solidify upon cooling, while
\textbf{thermosetting polymers} cure irreversibly. The cure may be induced by
heat, through a chemical reaction, or suitable irradiation. Thermoset materials
are usually liquid or malleable prior to curing and designed to be molded into
their final form, or used as adhesives. Once hardened a thermoset resin cannot
be reheated and melted to be shaped differently.
