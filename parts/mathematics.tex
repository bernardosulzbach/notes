\part{Mathematics}

\chapter{Significant figures}

\section{Definition}
The significant figures of a number are those digits that carry meaning
contributing to its precision. This includes all digits except all leading zeros
and trailing zeros when they are used to indicate the scale of the number.

\section{Calculations}
The result of addition and subtraction should have as many decimal places as the
term with less decimal places.

The result of multiplication and division should have as many significant
figures as the least precise term.

\section{Rounding}
If the first non-significant figure is a 5 not followed by any other digits or
followed only by zeros, rounding requires a tie-breaking rule.
\paragraph{Round half up} is the default rounding method implied in many
disciplines if not specified.
\paragraph{Round half to even} which rounds to the nearest even number.

\chapter{Set theory}

\section{Power set}
The power set \(P(S)\) is the set of all subsets of \(S\), including the empty
set and the set \(S\) itself.

It has \(2^n\) elements where \(n\) is the number of elements in \(S\).

\chapter{Linear algebra}

\section{Row operations}
\begin{itemize}
\item{Replace one row by the sum of itself and a multiple of another row.}
\item{Interchange two rows.}
\item{Multiply all entries in a row by a nonzero constant.}
\end{itemize}

\section{Row equivalence}
Two matrices are \textbf{row equivalent} if, and only if, there is a sequence
of row operations that transform one matrix into another.
Row equivalence is indicated by a tilde (\(\sim\)) between matrices.
\section{Echelon form}
A matrix is in \textbf{echelon form} if it has the shape resulting from a
Gaussian
elimination.
The following are properties of the echelon matrix:
\begin{itemize}
\item{All nonzero rows are above any rows of all zeros.}
\item{The leading entry of a row is at least one column to the right of the
leading entry of the row above it.}
\end{itemize}
The following are properties of the \textbf{reduced echelon matrix}:
\begin{itemize}
\item{The leading entry in any nonzero row is 1.}
\item{Each leading 1 is the only nonzero in its column.}
\end{itemize}

\section{Row reducing algorithm}
The \textbf{forward phase} of the row reducing algorithm produces the row
echelon form.
The \textbf{backward phase} produces the unique reduced row echelon form.
Computer algorithms usually select as a pivot the entry in a column that has the
largest value.
This strategy, called partial pivoting, is used to reduce roundoff errors in the
calculations.
Computers do not perform the backward phase in order to find the reduced echelon
form, instead,
they perform back substitution.
The best strategy is to use the reduced echelon form only when solving a system
by hand.
A flop is one arithmetic operation (+, -, *, /) on two real floating point
numbers.
(Traditionally, + and - were not considered to be flops, but nowadays they are).

\section{Solution existence and uniqueness}
A solution exists if there are no invalid equations, such as \( 0 = 1 \). The
solution is unique if, and only if, there are no free variables.

\subsection{Existence and uniqueness theorem}
A system is consistent if, and only if, the rightmost column of the augmented
matrix is not a pivot column.
Stated differently, a matrix equation of the form \(A\mathbf{x} = \mathbf{b}\)
has a solution if, and only if, \(\mathbf{b}\) is a linear combination of the
columns of \(\mathbf{A}\), that is, if \(\mathbf{b} \in \mathrm{Span}
\{\mathbf{a}_1, \ldots, \mathbf{a}_n\}\).

\section{Parallelogram rule for addition}
If \(\mathbf{u}\) and \(\mathbf{v}\) in \(\mathbb{R}^2\)
are represented as points in the plane,
then \(\mathbf{u} + \mathbf{v}\) corresponds to the fourth vertex of the
parallelogram whose other vertices are \(\mathbf{0}\), \(\mathbf{u}\) and
\(\mathbf{v}\).

\section{Linear combinations}
Given vectors \(\mathbf{v}_1, \mathbf{v}_2, \ldots, \mathbf{v}_p\) in
\(\mathbb{R}^n\) and given scalars \(c_1, c_2, \ldots, c_p\) in \(\mathbb{R}\),
the vector \(\mathbf{y}\) defined by
\[ \mathbf{y} = c_1 \mathbf{v}_1 + \ldots + c_p \mathbf{v}_p \]
is called a \textbf{linear combination} of
\(\mathbf{v}_1, \ldots, \mathbf{v}_p\) with weights \(c_1, \ldots, c_p\).

\section{Span}
The set of all linear combinations of \(\mathbf{v}_1, \ldots, \mathbf{v}_p\) in
\(\mathbb{R}^n\) is called the \textbf{subset of}\(\mathbb{R}^n\) spanned (or
generated) by \(\mathbf{v}_1, \ldots, \mathbf{v}_p\) and is denoted by

\(\mathrm{Span}\{\mathbf{v}_1, \ldots, \mathbf{v}_p\}\). If a matrix \(A\) has a
pivot in every row, then \(\mathrm{Span} \{\mathbf{a}_1, \ldots, \mathbf{a}_n\}
= \mathbb{R}^m\).

\section{Matrix equation solution}
If \(A\) is a \(m \times n\) matrix, with columns \(\mathbf{a}_1, \ldots,
\mathbf{a}_n\) and \(\mathbf{b}\) is in \(\mathbb{R}^m\),
the matrix equation
\[ A\mathbf{x} = \mathbf{b} \]
has the same solution as the vector equation
\[ x_1\mathbf{a}_1 + \ldots + x_n\mathbf{a}_n = \mathbf{b} \]
which, in turn,
has the same solution as the system of linear equations whose
augmented matrix is
\[ [ \mathbf{a}_1 \ldots \mathbf{a}_n \mathbf{b} ] \]

\section{Solution sets of linear systems}
A system of linear equations is \textbf{homogeneous}
if it can be written in the form \( \mathbf{A x} = \mathbf{0} \).
All homogeneous systems have a \textbf{trivial solution} that is \(\mathbf{0}\).
If the system has at least one free variable, it has a nontrivial solution.

\begin{theorem}
  Suppose that \( \mathbf{Ax} = \mathbf{b} \) is consistent for some given
  \(\mathbf{b}\), let \(\mathbf {p}\) be a solution. Then the solution set of
  \( \mathbf{Ax} = \mathbf{b} \) is the set of all vectors of the form
  \( \mathbf{w} = \mathbf{p} + \mathbf{v} \), where \( \mathbf{v} \) is any
  solution of the homogeneous equation \( \mathbf{Ax} = \mathbf{0} \).
\end{theorem}

\section{Solution of a system in parametric form}
\begin{enumerate}
\item{Row reduce the augmented matrix to reduced row echelon form.}
\item{Express each basic variable in terms of any free variables.}
\item{Write \(\mathbf{x}\) as a vector whose entries depend on the free
variables.}
\item{Decompose \(\mathbf{x}\) into a linear combination of vectors.}
\end{enumerate}

\subsection*{Solved problem}
Write the general solution of \( 10 x_1 - 3 x_2 - 2 x_3 = 7\)
in parametric vector form.
\[ \left[ \begin{array}{cccc}
10 & -3 & -2 & 7 \end{array} \right] \sim
\left[ \begin{array}{cccc}
1 & -.3 & -.2 & .7 \end{array} \right] \]
    \[ x_1 = .7 + .3x_2 + .2x_3 \]
  \[ \mathbf{x} = \left[ \begin{array}{c} .7 \\ 0 \\ 0 \end{array} \right]
+ x_2 \left[ \begin{array}{c} .3 \\ 1 \\ 0 \end{array} \right]
+ x_3 \left[ \begin{array}{c} .2 \\ 0 \\ 1 \end{array} \right] \]

\section{Applications of linear systems}
In this section, it is shown how linear systems with multiple solutions can
arise naturally.
\subsection{Leontief model}
Also known as the Leontief ``production'' model.
If one knows the total output of a sector for a certain period of time and how
it is
divided among other sectors, then it is proven that
\begin{quote}
There exist equilibrium prices that can be assigned to the total outputs of the
various sectors in such a way that the income of each sector exactly balances
its
expenses.
\end{quote}

Suppose the following model is a decent representation of an economic.
\begin{center}
\begin{tabular}{| c c c | l |}
\hline
Coal & Electric & Steel & Purchased by \\
\hline
.0 & .4 & .6 & Coal \\
.6 & .1 & .2 & Electric \\
.4 & .5 & .2 & Steel \\
\hline
\end{tabular}
\end{center}

Then it is known that the equilibrium price of coal \(p_C = .4p_E + .6p_S\).
Therefore, \(p_C - .4p_E - .6p_S = 0\). Similar equations can be obtained from
the other two lines. At the end of the process, the linear system is

\[
\begin{bmatrix}
1   & -.4 & -.6 & 0 \\
-.6 &  .9 & -.2 & 0 \\
-.4 & -.5 &  .8 & 0
\end{bmatrix}
\]

By solving this system, it is known that

\[
p = \begin{bmatrix} p_C \\ p_E \\ p_S \end{bmatrix}
= \begin{bmatrix} .94 p_S \\ .85 p_S \\ p_S \end{bmatrix}
\]

Any (nonnegative) choice for \(p_S\) results in correct equilibrium prices.

\subsection{Balancing chemical equations}
Chemical equations describe the consumption of reactants and the formation
of products in chemical reactions.
Simple linear systems may be used to balance chemical reactions.
The burning of butanol is represented by:
\[
\left( x_1 \right) \text{C}_4\text{H}_{10}\text{O} +
\left( x_2 \right) \text{O}_2 \rightarrow
\left( x_3 \right) \text{CO}_2 +
\left( x_4 \right) \text{H}_2\text{O}
\]
In order to balance this reaction,
one must find the coefficients that make both sides have the same amount of
carbon,
hydrogen and oxygen.
This problem can be rewritten as:
\[
x_1 \begin{bmatrix} 4 \\ 10 \\  1\end{bmatrix} +
x_2 \begin{bmatrix} 0 \\  0 \\  2\end{bmatrix} +
x_3 \begin{bmatrix}-1 \\  0 \\ -2\end{bmatrix} +
x_4 \begin{bmatrix} 0 \\ -2 \\ -1\end{bmatrix} =
\begin{bmatrix}0 \\ 0 \\ 0\end{bmatrix}
\]
Row reduction of the corresponding augmented matrix gives us
\[
\textbf{p} = \begin{bmatrix}.2x_4 \\ 1.2x_4 \\ .8x_4 \\ x_4\end{bmatrix}
\]
The smallest selection of \(x_4\) that makes all coefficients integers is \(5\).
This makes
\[
  \text{C}_4\text{H}_{10}\text{O} +
6 \text{O}_2 \rightarrow
4 \text{CO}_2 +
5 \text{H}_2\text{O}
\]
a balanced chemical reaction.

\subsection{Network flow}
Systems of linear equations arise when scientists study the flow of something
through a network.
For instance, traffic engineers monitor the pattern of traffic flow in a grid of
city streets.
Electrical engineers calculate current flow through circuits.
And economists analyze the distribution of products from manufacturers to
consumers.
A network consists of a set of points (called junctions or nodes),
with lines or arcs called branches connecting some or all of the junctions.
The direction of flow in each branch is indicated,
and the flow amount (rate) is either shown or is denoted by a variable.
The basic assumption of network flow is that the total flow into the network
equals
the total flow out of the network and that the total flow into a junction equals
the total
flow out of the junction.
In a similar fashion, the flow at each junction is described by a linear
equation. The
problem of network analysis is to determine the flow in each branch when partial
information (such as the flow into and out of the network) is known.

\subsection{Solved exercises}
\subsubsection{Introduction}
This exercises are from the fourth edition of Linear Algebra and its
Applications.
\subsubsection{Leontief model}
\paragraph{1)}
An economy has four sectors: Agriculture, Manufacturing, Services, and
Transportation.
Agriculture sells 20\% of its output to Manufacturing, 30\% to Services, 30\% to
Transportation, and retains the rest.
Manufacturing sells 35\% of its output to Agriculture, 35\% to Services, 20\% to
Transportation, and retains the rest.
Services sells 10\% of its output to Agriculture, 20\% to Manufacturing, 20\% to
Transportation, and retains the rest.
Transportation sells 20\% of its output to Agriculture, 30\% to Manufacturing,
20\% to Services, and retains the rest.

\begin{enumerate}
\item Construct the exchange table for this economy.
\item Find a set of equilibrium prices for the economy if the value of
  Transportation is \(\$10.00\) per unit.
\item {The Services sector launches a successful ``eat farm fresh'' campaign,
       and increases its share of the output from the Agricultural sector to
40\%,
       whereas the share of Agricultural production going to Manufacturing falls
to 10\%.
       Construct the exchange table for this new economy.}
\item {Find a set of equilibrium prices for this new economy if the value of
    Transportation is still \(\$20.00\) per unit.
       What effect has the ``eat farm fresh'' campaign had on the equilibrium
prices for the sectors in this economy?}
\end{enumerate}

\subparagraph{Solution}

\begin{enumerate}
  \item {
  \[ \begin{bmatrix}
.2 & .35 & .1 & .2 \\
.2 &  .1 & .2 & .3 \\
.3 & .35 & .5 & .2 \\
.3 &  .2 & .2 & .3
      \end{bmatrix} \]}
  \item{\(p_A = 7.99\), \(p_M = 8.36\), \(p_S = 14.65\), and \(p_T = 10.00\).}
  \item {\[ \begin{bmatrix}
.2 & .35 & .1 & .2 \\
.1 &  .1 & .2 & .3 \\
.4 & .35 & .5 & .2 \\
.3 &  .2 & .2 & .3
            \end{bmatrix} \]}
  \item{\(p_A = 7.81\), \(p_M = 7.67\), \(p_S = 15.62\), and \(p_T = 10.00\).}
  \item{Therefore, the campaign has benefited Services.}
\end{enumerate}

\section{Matrices}
\subsection{Properties of the determinant}
\begin{enumerate}
  \item{\(\det(I)=1\).}
  \item{\(\det(A)=\det(A^{T})\).}
  \item{If \(B\) is \(A\) with a row multiplied by \(k\),
      then \(\det(B)=k\det(A)\).}
  \item{If \(B\) is \(A\) with two rows swapped, then \(\det(B)=-\det(A)\).}
  \item{If two or more rows of \(A\) are identical, then \(\det(A)=0\).}
  \item{If the elements above or below the diagonal of \(A\) are \(0\), then
      \(\det(A)=\prod_{i=1}^{m}{A_{i,i}}\).}
  \item{If the elements of \(A\), \(B\), and \(C\) are equal,
      except for one row of C that is equal to the sum of the two corresponding
      rows of \(A\) e \(B\), then
      \(\det{\left(C\right)}=\det{\left(A\right)}+\det{\left(B\right)}\).}
  \item{For square matrices \(A\) and \(B\) of equal size,
      \(\det{\left(AB\right)}=\det{\left(A\right)}\det{\left(B\right)}\).}
  \item{If \(B\) is \(A\) with one row multiplied by a nonzero constant added
      to a parallel row, then then \(\det(A)=\det(B)\).}
  \item{\(\det{\left(A^{-1}\right)}=\frac{1}{\det{\left(A\right)}}\).}
\end{enumerate}

\subsection{Inverse matrix}

\begin{proof}
The inverse matrix is unique.

Let \(B\) and \(C\) be the inverse matrices of \(A\).
\[B = BI = B(AC) = (BA)C = IC = C\]
\end{proof}

\subsubsection{Inverse matrix by Gauss-Jordan}
\[ \left[\mathrm{A} \mid \mathrm{I} \right] \rightarrow
\left[\mathrm{I} \mid \mathrm{A} ^ {-1} \right] \]

The adjoint matrix is the transposed of the cofactor matrix.
\[A^{-1} = \frac{\mathrm{adj}(A)}{\mathrm{D}} =
\frac{(\mathrm{cof}(A))^{T}}{\mathrm{D}}\]

\subsection{Singular matrix}
A square matrix that does not have a matrix inverse. A matrix is singular if,
and only if, its determinant is 0.


\chapter{Series}

\section{Arithmetic series}
An arithmetic series is the sum of a sequence in which each term is computed
from the previous one by adding (or subtracting) a constant.
\[a_n = a_1 + (n-1)r\]
\[S_{p..q} = \frac{(q - p + 1)(a_q + a_p)}{2}\]
Note that this expression is just the product of the number of terms by the
average value.

\section{Geometric series}
\[a_n = a_1 q^{(n-1)}\]
\[S_n = a_1 \frac{q^n - 1}{q-1}\]

\chapter{Complex numbers}

\section{Notation}
\[\cis \left ( \argument \right ) =  \cos \left ( \argument \right ) + i \sin
\left ( \argument \right )\]

\section{Formulas}
\[a + bi = r \left( \cos \left ( \argument \right ) + i \sin \left ( \argument
\right ) \right )
= r \cis \left ( \argument \right )\]
\[\conj{z} = a - bi = r \cis \left (- \argument \right)\]
\[r = \sqrt{a^2 + b^2}\]
\[\phi = \tan^{-1} \left ( \frac{b}{a} \right)\]

\[z_1 z_2 = r_1 r_2 \left ( \cis \left ( \phi_1 + \phi_2 \right ) \right )\]
\[\frac{z_1}{z_2} = \frac{r_1}{r_2} \left ( \cis \left ( \phi_1 - \phi_2 \right
) \right )\]
\[z^n = r^n \cis \left ( n \argument \right )\]
\[z^{\reciprocal{n}} = r^{\reciprocal{n}} \cis \left ( \frac{\argument + 2 \pi
k}{n} \right ), \quad
k : k \in \integers \text{ and } 0 \le k \le n - 1\]

Euler's Formula
\[e^{ix} = \cis \left ( x \right)\]

\chapter{Logarithms}

\paragraph{Logarithm.} The logarithm is the inverse operation to exponentiation.
The natural logarithm of \(t\) equals the integral of \(x^{-1}\) in respect to x
from \(1\) to \(t\).

\section{Definition}

\[y = \log_a x \iff a^y = x, \quad a > 0, a \ne 0\]
\[\log_a 0 = \begin{cases} -\infty, \quad a > 1 \\ +\infty, \quad a < 1
\end{cases}\]
\[\log x y = \log x + \log y\]
\[\log \frac{x}{y} = \log x - \log y\]
\[\log x^n = n \log x\]
\[\log_a b = \frac{\log_c b}{\log_c a} = \left ( \log_a c \right ) \left (
\log_c b \right )\]
\[\log a_b = \reciprocal{\log_b a}\]
\[e=\lim_{k\to\infty} \left ( 1 + \reciprocal{k} \right ) ^ k\]

\chapter{Probability and statistics}

\section{Discrete distributions}

\subsection{Geometric distribution}
\subsubsection*{Probability mass function}
\[P(X = k) = (1 - p)^{k - 1}p\]
\subsubsection*{Tail probabilities function}
\[P(X \ge k) = (1 - p)^{k - 1}\]
This probability distribution has mean \(\frac{1}{p}\) and variance \(\frac{1 - p}{p^2}\).

\paragraph{Problem.}
Suppose that the probability of an item on an assembly
line being defective is \(p = 0.02\). What is the average number of items
inspected before finding the first defective item?
\paragraph{Solution.}
\[\frac{1}{p} = \frac{1}{0.02} = 50\]

\paragraph{Problem.}
What is the probability that none of the first ten sampled
items are not defective?
\paragraph{Solution.}
\[P(X \ge 11) = (1 - p)^{10} = 0.98^{10} \approx 0.817\]

\subsection{Poisson distribution}
\subsubsection*{Probability mass function}
\[P(X = k) = e^{-\lambda}\frac{\lambda^k}{k!}\]

\paragraph{Problem.}
This probability distribution has mean \(\lambda\) and variance \(\lambda\).
Suppose that the number of creatures of a given species that we expect to find
in a one square meter follows the Poisson(5) distribution. What is the
probability of finding exactly 7 creatures of this species?
\paragraph{Solution.}
\[P(X = 7) = e^{-5}\frac{5^7}{7!} = e^{-5}\frac{78125}{5040} \approx 0.104\]

\paragraph{Problem.}
For a Poisson(\(\lambda\)) distribution, suppose that the probability
attached to 0 is 0.02949. Determine \(\lambda\).
\paragraph{Solution.}
\begin{align*}
P(X = 0) = e^{-\lambda}\frac{\lambda^0}{0!} &= e^{-\lambda} = 0.02949 \\
& \therefore -\lambda = \ln \left(0.02949\right)\\
& \therefore \lambda \approx 3.524
\end{align*}

\section{Linear relationship}
A linear relationship between \(y\) and \(x\) is denoted by:
\[\mu_{y | x} = \beta_0 + \beta_1 x + \epsilon\]
where
\begin{itemize}
  \item[\(\mu_{y | x}\)] is the true mean of y for a given value of x
  \item[\(\beta_0 + \beta_1 x\)] is a line
  \item[\(\epsilon\)] is a random error
\end{itemize}

The estimated relation line:
\[\hat y = \hat {\beta_0} + \hat {\beta_1} x\]

\section{Pythagorean means}
The three classical Pythagorean means are the arithmetic mean (A), the
geometric mean (G), and the harmonic mean (H). They are defined by:
\[A = \frac{\sum{x}}{n}\]
\[G = \left( \prod{x} \right) ^ \reciprocal{n}\]
\[H = \frac{n}{\sum{\reciprocal{x}}}\]

If all \(x\) are positive, the following relation holds
\[min \le H \le G \le A \le max\]

\section{Measures of statistical dispersion}
Variance
\[\sigma^2 = \sum \frac{\left ( k-\mean{x} \right ) ^ 2}{n}\]
Standard Deviation
\[\sigma = \sqrt{\sum \frac{\left ( k-\mean{x} \right ) ^ 2}{n}}\]

\section{Normal distribution}
The normal distribution is a common continuous probability distribution denoted
by \(N(\mu, \sigma)\). \(\mu\) is the mean of the distribution (and also its
median and mode) and \(\sigma\) is its standard deviation.

If \(\mu = 0\) and \(\sigma = 1\), the distribution is called the
\textbf{standard normal distribution}.

\section{Fifty challenging problems in probability}

\begin{Exercise}
A drawer contains red socks and black socks. When two socks are drawn at random, the probability that both are red is \(\frac{1}{2}\).
\begin{enumerate}[a)]
\item How small can the number of socks in the drawer be?
\item How small if the number of black socks is even?
\end{enumerate}
Taken from \cite[p.~1]{fifty-challenging-problems-1987}.
\end{Exercise}
\begin{Answer}
\begin{enumerate}[a)]
\item We know the following.
\[\frac{r}{r + b} \times \frac{r - 1}{r + b - 1} = \frac{1}{2}\]
\[\frac{r}{r + b} > \frac{r - 1}{r + b - 1}\]
\begin{equation}
\left(\frac{r}{r + b}\right)^2 > \frac{1}{2} > \left(\frac{r - 1}{r + b - 1}\right)^2 \label{fifty-challenging-problems-equation-1-1}
\end{equation}
From \ref{fifty-challenging-problems-equation-1-1} we can get the following inequality.
\begin{equation}
\left(\sqrt{2} + 1\right) b + 1 > r > \left(\sqrt{2} + 1\right) b
\end{equation}
By letting \(b = 1\), the only possible integer value of \(r\) is 3.
Testing with \(r = 3\) shows us that the smallest possible number of socks is 4.
\item For even values of \(b\), the same analysis shows us that 21 socks is the smallest possible number of socks.
\end{enumerate}
The book points out that this is a problem in number theory as a result of Diophantine Analysis from Pell's equation.
\end{Answer}

\begin{Exercise}
Elmer's father offers him a prize if he wins (at least) two tennis sets in a row in a three-set series to be played with his father and the club champion alternately: father-champion-father or champion-father-champion, according to Elmer's choice.
The champion is a better player than Elmer's father. What series should Elmer choose?

Taken from \cite[p.~1]{fifty-challenging-problems-1987}.
\end{Exercise}
\begin{Answer}
\[P_i = P_f P_c P_f + P_f P_c (1 - P_f) + (1 - P_f) P_c P_f
= P_f P_c (2 - P_f)\]
\[P_{ii} = P_c P_f P_c + P_c P_f (1 - P_c) + (1 - P_c) P_f P_c
= P_c P_f (2 - P_c)\]

Because \(P_f > P_c\), \(P_{ii} > P_i\), and Elmer should choose champion-father-champion.

From the book, the importance of winning the middle game outweighs the disadvantage of playing the champion twice.
\end{Answer}

\begin{Exercise}
A three-man jury has two members each of whom independently has a probability \(p\) of making the correct decision and a third member who flips a coin for each decision.
A one-man jury has a probability \(p\) of making the correct decision.
Which jury has the better probability of making the correct decision?

Taken from \cite[p.~1]{fifty-challenging-problems-1987}.
\end{Exercise}
\begin{Answer}
\[P_1 = p\]
\[P_3 = p^2 + 2 p (1 - p) / 2 = p^2 + p - p^2 = p\]

So both juries have the same probability of making the correct decision.
\end{Answer}

\chapter{Geometry}

\section{Trigonometry}

For a right triangle with sides a and b, hypotenuse c
\[\sin{\alpha} = \frac{a}{c}\]
\[\cos{\alpha} = \frac{b}{c}\]
\[\tan{\alpha} = \frac{a}{b}\]
\[\csc{\alpha} = \reciprocal{\sin{\alpha}} = \frac{c}{a}\]
\[\sec{\alpha} = \reciprocal{\cos{\alpha}} = \frac{c}{b}\]
\[\cot{\alpha} = \reciprocal{\tan{\alpha}} = \frac{b}{a}\]

Sums and differences of sines, cosines, and tangents
\[{\sin{\left( \alpha \pm \beta \right)} =
  \sin{\alpha}\cos{\beta} \pm \sin{\beta}\cos{\alpha}}\]
\[{\cos{\left( \alpha \pm \beta \right)} =
  \cos{\alpha}\cos{\beta} \mp \sin{\alpha}\sin{\beta}}\]
\[{\tan{\left( \alpha \pm \beta \right)} =
  \frac{\tan{\alpha} \pm \tan{\beta}}{1 \mp \tan{\alpha} \tan{\beta}}}\]

\section{Right triangle}
For a right triangle with sides a and b, hypotenuse c, projections m and n (of
a and b, respectively), and height h,
the following expressions are true

\[a^2 = m c\]
\[b^2 = n c\]
\[h^2 = m n\]
\[a b = c h\]

\section{Equilateral triangle}
\[h = \frac{a \sqrt{3}}{2}\]
\[r = \frac{a \sqrt{3}}{6}\]
\[R = \frac{a \sqrt{3}}{3}\]
\[A = \frac{a^2 \sqrt{3}}{4}\]

\section{All triangles}
\[A = a b \sin \gamma \]
\[A = p r\]
\[A = \sqrt{p \left ( p - a \right ) \left ( p - b \right ) \left ( p - c
\right )}\]

where

\begin{itemize}
  \item[\(p\)] is the semiperimeter;
  \item[\(r\)] is the inradius (the radius of the inscribed circle).
\end{itemize}

\paragraph{Median.} In geometry, a median of a triangle is a line segment
joining a vertex to the midpoint of the opposing side. Every triangle has
exactly three medians, one from each vertex. In the case of isosceles and
equilateral triangles, a median bisects any angle at a vertex whose two adjacent
sides are equal in length.

\paragraph{Relation to center of mass.} Each median of a triangle passes through
the triangle's centroid, which is the center of mass of an object of uniform
density in the shape of the triangle. Thus the object would balance on the
intersection point of the medians.

\paragraph{Each median divides the area of the triangle in half.}

\section{Law of sines}
\[\frac{a}{\sin{A}} = \frac{b}{\sin{B}} = \frac{c}{\sin{C}} = 2R\]
where \(a\), \(b\), and \(c\) are the lengths of the sides of a triangle, and
\(A\), \(B\), and \(C\) are the opposite angles, and \(R\) is the radius of
the triangle's circumcircle.

\section{Law of cosines}
\[a^2 = b^2 + c^2 - 2bc\cos{A}\]

\section{Quadrilateral}
\[A = \frac{d_1 d_2 \sin{\phi}}{2}\]

where

\begin{itemize}
  \item[\(d\)] is a diagonal;
  \item[\(\phi\)] is the angle between the diagonals.
\end{itemize}

\section{Regular hexagon}

\[r = m = \frac{l \sqrt{3}}{2}\]
\[A = pr = 3 l r = \frac{3 l^2 \sqrt{3}}{2}\]

\section{Circle}
\paragraph{Intersecting Chord Theorem.} When two chords intersect each
other inside a circle, the products of their segments are equal.
\[a_1 a_2 = b_1 b_2\]

\section{Ellipse}
In mathematics, an ellipse is a curve on a plane that surrounds two focal points
such that the sum of the distances to the two focal points is constant for every
point on the curve.

An ellipse with center \(C \left( x_0, y_0 \right)\) is represented by
\[\frac{\left(x - x_0\right)^2}{a^2}+\frac{\left(y - y_0\right)^2}{b^2}=1\]
or
\[\frac{\left(x - x_0\right)^2}{b^2}+\frac{\left(y - y_0\right)^2}{a^2}=1\]
\subsection{Focus}
The focal distance \(c\) is given by the Pythagorean theorem:

\[a^2=b^2+c^2\]

\subsection{Area}
The area of an ellipse is:

\[A = \pi a b\]

\subsection{Eccentricity}
The eccentricity of an ellipse, usually denoted by \(\epsilon\), is the ratio of
the distance between the two foci, to the length of the major axis:

\[\epsilon = \frac{2f}{2a} = \frac{f}{a}\]

\section{Pyramids}
\[V = \frac{bh}{3}\]

\chapter{Equations}
\section{Discriminant}
\[\Delta = b^2 - 4ac\]

\section{Vieta's formulas}
A polynomial \(P(x) = a_n x^n + a_{n-1} x^{n-1} + ... + a_0\), (with the
coefficients being real or complex numbers and \(a_n \neq 0\)) is known by the
fundamental theorem of algebra to have \(n\) (not necessarily distinct) complex
roots \(x_1, x_2, ..., x_n\).

Vieta's formulas relate the polynomial's coefficients \(a_k\) to signed sums
and products of its roots \(x_i\) as follows:

\[\sum_{1 \le i \le n} x_i = - \frac{a_{n-1}}{a_n}\]
\[\sum_{1 \le i_1 < i_2 \le n} x_{i_1} x_{i_2} = \frac{a_{n-2}}{a_n}\]
\[\sum_{1 \le i_1 < i_2 < i_3 \le n} x_{i_1} x_{i_2} x_{i_3} = -
\frac{a_{n-3}}{a_n}\]
\[\sum_{1 \le i_1 < i_2 < ... < i_k \le n} x_{i_1} x_{i_2} ... x_{i_k} = (-1)^k
\frac{a_{n-k}}{a_n}\]

\chapter{Calculus}
\section{Limits}
\subsection{Limits with multiple variables}
If the function is continuous where you are trying to find the limit, simply
plug in the variables. If this is not possible, the squeeze theorem might be a
better solution. Lastly, you should try to prove that the limit does not exist.

When proving that the limit of a function of one variable does not exist, you
would test if the limits from the right and from the left were different.
When dealing with functions of multiple variables, if the limits along any two
different smooth curves are different, the limit you are trying to find does
not exist. Usually, \(x = 0\), \(y = 0\) and \(x = y\) are good choices when
proving that a limit does not exist.

\subsection{Limits with multiple variables exercises}

\paragraph{Problem.}
Show that the following limit does not exist.
\[\lim_{\left(x, y\right) \rightarrow \left(0, 0\right)} \frac{x + y}{2x^2 + y^2}\]
\paragraph{Solution.}
We look at the limit as \(y \rightarrow 0\) by the smooth curve \(x = 0\).
\[\lim_{y \rightarrow 0} \frac{x + y}{2x^2 + y^2}
= \lim_{y \rightarrow 0} \frac{y}{y^2}
= \lim_{y \rightarrow 0} \frac{1}{y} = +\infty\]
Therefore, the limit does not exist.

\paragraph{Problem.}
Evaluate the following limit.
\[\lim_{\left(x, y\right) \rightarrow \left(0, 0\right)} \frac{\sin \left(x^2 + y^2\right)}{x^2 + y^2}\]
\paragraph{Solution.}
Let \(z = x^2 + y^2\), then
\[\lim_{\left(x, y\right) \rightarrow \left(0, 0\right)} \frac{\sin \left(x^2 + y^2\right)}{x^2 + y^2}
= \lim_{z \rightarrow 0^+} \frac{\sin \left(z\right)}{z}
= 1\]
by the squeeze theorem or by l'Hôpital's rule.

\paragraph{Problem.}
Evaluate the following limit.
\[\lim_{\left(x, y\right) \rightarrow \left(0, 0\right)} e^{-1/\left(x^2 + y^2\right)}\]
\paragraph{Solution.}
Let \(z = x^2 + y^2\), then
\[\lim_{\left(x, y\right) \rightarrow \left(0, 0\right)} e^{-1/\left(x^2 + y^2\right)}
= \lim_{z \rightarrow 0^+} e^{-z^{-1}}\]
Because \[\lim_{z \rightarrow 0^+} -z^{-1} = -\infty\]
we know that \[\lim_{z \rightarrow 0^+} e^{-z^{-1}} = \lim_{x \rightarrow -\infty} e^{x} = 0\]

\paragraph{Problem.}
Determine if the following limit exists. If so, find its value.
\[\lim_{\left(x, y\right) \rightarrow \left(0, 0\right)} \frac{x^4 - y^4}{x^2 + y^2}\]
\paragraph{Solution.}
First we simplify the expression.
\[\frac{x^4 - y^4}{x^2 + y^2} = \frac{\left(x^2 - y^2\right) \left(x^2 + y^2\right)}{x^2 + y^2} = x^2 - y^2\]
Then we find the limit of the simplified expression. Clearly, this limit does
exist because \(x^2 - y^2\) is continuous everywhere.
\[\lim_{\left(x, y\right) \rightarrow \left(0, 0\right)}{\frac{x^4 - y^4}{x^2 + y^2}}
= \lim_{\left(x, y\right) \rightarrow \left(0, 0\right)} x^2 - y^2 = 0\]

\section{Derivatives}
\[\frac{\dif}{\dif x} \sin x = \cos x\]
\[\frac{\dif}{\dif x} \cos x = -\sin x\]

\subsection{Identities}
\[\frac{\dif}{\dif x} \ln x = \reciprocal{x}\]
\textbf{Chain Rule}
\[\left\{f\left[g\left(x\right)\right]\right\}'
= f'\left[g\left(x\right)\right] g'\left(x\right)\]

\subsection{Partial derivatives}
A partial derivative is a derivative of a function of two or more variables
with respect to a single variable while the others are treated as constants.
The partial derivative of a function \(f\) with respect to \(x\) is indicated by
\(\frac{\partial f}{\partial x}\).

Some authors call \(\frac{\partial f}{\partial x}\left(x_0, y_0\right)\) the
slope of the surface in the x-direction at \(\left(x_0, y_0\right)\).
Similarly, \(\frac{\partial f}{\partial y}\left(x_0, y_0\right)\) is the slope
of the surface in the y-direction at \(\left(x_0, y_0\right)\).

\paragraph{Problem.}
Find the slope of the surface \(f(x, y) = x^{3}y^{2} + 5y\) in the y-direction at the point \((1, -2)\).
\paragraph{Solution.}
First, we differentiate with respect to \(y\) to find an expression for the slope we are interested in.
\[\frac{\partial f}{\partial y} = 2x^{3} + 5\]
Then we evaluate the expression at the point of interest.
\[\frac{\partial f}{\partial y}\left(1, -2\right) = 2 + 5 = 7\]
Therefore, the slope of the surface \(f(x, y) = x^{3}y^{2} + 5y\) in the y-direction at the point \((1, -2)\) is 7.

\subsubsection{Higher-Order Partial Derivatives}
The partial derivatives of partial derivatives are called second-order partial
derivatives. The second-order derivatives for a function \(f\left(x, y\right)\)
are
\[\frac{\partial^2 f}{\partial x \partial x} = f_{xx},
\quad \frac{\partial^2 f}{\partial x \partial y} = f_{yx},
\quad \frac{\partial^2 f}{\partial y \partial x} = f_{xy},
\quad \frac{\partial^2 f}{\partial y \partial y} = f_{yy}\]
Notice how these notations have different conventions for the order of
differentiation. The \(\partial\) notation being evaluated from right to left
and the subscript notation being evaluated from left to right.

The second-order partials \(f_{xy}\) and \(f_{yx}\) are called mixed partials.

\begin{theorem}
    Let \(f\) be a function of two variables. If \(f_{xy}\) and \(f_{yx}\) are
    continuous on some open disk, then \(f_{xy} = f_{yx}\) on that disk.
\end{theorem}

\paragraph{Problem.} Let \(f\left(x, y\right) = x^2 e^x e^y + xy\). Find \(f_{xxy}\).
\paragraph{Solution.}
\begin{align*}
    &\frac{\partial^3}{\partial y \partial x \partial x} \left(x^2 e^x e^y + xy\right) \\
    &=\frac{\partial^2}{\partial y \partial x} \left(\frac{\partial}{\partial x} \left(x^2 e^x e^y + xy\right)\right) \\
    &=\frac{\partial^2}{\partial y \partial x} \left(2x e^x e^y + x^2 e^x e^y + y\right) \\
    &=\frac{\partial}{\partial y} \left(\frac{\partial}{\partial x} \left(2x e^x e^y + x^2 e^x e^y + y\right)\right) \\
    &=\frac{\partial}{\partial y} \left(\frac{\partial}{\partial x} \left(\left(2x + x^2\right) e^x e^y\right)\right) \\
    &=\frac{\partial}{\partial y} \left(\left(2x + x^2\right) e^x e^y + \left(2 + 2x\right) e^x e^y\right) \\
    &=\frac{\partial}{\partial y} \left(\left(x^2 + 4x + 2\right) e^x e^y\right) \\
    &= \left(x^2 + 4x + 2\right) e^x e^y
\end{align*}

\subsubsection{Local Linear Approximations}
For a function \(f\left(x, y\right)\) We call
\[L\left(x, y\right) = f\left(x_0, y_0\right) + f_x\left(x_0, y_0\right)\left(x - x_0\right) + f_y\left(x_0, y_0\right)\left(y - y_0\right)\]
the local linear approximation of \(f\) at \(\left(x_0, y_0\right)\).

\subsubsection{Chain Rule for Partial Derivatives}
For a function \(z\left(x, y\right)\), where \(x = x\left(u, v\right)\) and \(y = y\left(u, v\right)\).
\[\frac{\partial z}{\partial u} = \frac{\partial z}{\partial x}\frac{\partial x}{\partial u} + \frac{\partial z}{\partial y}\frac{\partial y}{\partial u}\]

Similar formulas may be derived for functions of three or more variables.

\paragraph{Problem.} Let
\[w = e^{xyz}, \quad x = 3u + v, \quad y = 3u - v, \quad z = u^2v.\]
Use the chain rule to find \(\frac{\partial w}{\partial u}\). \cite{anton-bivens-davis}
\paragraph{Solution.}
\begin{align*}
    \frac{\partial w}{\partial u}
    &= \frac{\partial w}{\partial x}\frac{\partial x}{\partial u}
    + \frac{\partial w}{\partial y}\frac{\partial y}{\partial u}
    + \frac{\partial w}{\partial z}\frac{\partial z}{\partial u} \\
    &= 3 yz e^{xyz} + 3 xz e^{xyz} + 2 xyuv e^{xyz} \\
    &= e^{xyz} \left(3 yz + 3 xz + 2 xyuv\right)
\end{align*}

\paragraph{Problem.} Let
\[w = x^2 + y^2 - z^2, \quad x = \rho \sin \phi \cos \theta, \quad y = \rho \sin \phi \sin \theta, \quad z = \rho \cos \phi.\]
Use the chain rule to find \(\frac{\partial w}{\partial \rho}\). \cite{anton-bivens-davis}
\paragraph{Solution.}
\begin{align*}
    \frac{\partial w}{\partial \rho}
    &= \frac{\partial w}{\partial x}\frac{\partial x}{\partial \rho}
    + \frac{\partial w}{\partial y}\frac{\partial y}{\partial \rho}
    + \frac{\partial w}{\partial z}\frac{\partial z}{\partial \rho} \\
    &= 2 x \sin \phi \cos \theta
    + 2 y \sin \phi \sin \theta
    - 2 z \cos \phi \\
    &= 2 \rho \sin ^2 \phi \cos ^2 \theta
    + 2 \rho \sin ^2 \phi \sin ^2 \theta
    - 2 \rho \cos ^2 \phi \\
    &= 2 \left( \rho \sin ^2 \phi \cos ^2 \theta + \rho \sin ^2 \phi \sin ^2 \theta - \rho \cos ^2 \phi \right) \\
    &= 2 \left( \left( \rho \sin ^2 \phi \right) \left( \cos^2 \theta + \sin ^2 \theta \right) - \rho \cos ^2 \phi \right) \\
    &= 2 \left( \rho \sin ^2 \phi - \rho \cos ^2 \phi \right) \\
    &= 2 \rho \left( \sin ^2 \phi - \cos ^2 \phi \right) \\
    &= - 2 \rho \cos 2 \phi
\end{align*}

\section{Series}
\section{Series for e}
\[e^x = \frac{x^0}{0!} + \frac{x^1}{1!} + \frac{x^2}{2!} + \frac{x^3}{3!} + \ldots\]

\subsection{Intuition behind this series}
The following demonstration should help one understand why the abovementioned
series converges to \(e\).

\begin{align*}
    \frac{\dif}{\dif x} e^x
    &= \frac{\dif}{\dif x} \left( \frac{x^0}{0!} + \frac{x^1}{1!} + \frac{x^2}{2!} + \frac{x^3}{3!} + \ldots \right) \\
    &= \frac{0 x^{-1}}{0!} + \frac{1 x^0}{1!} + \frac{2 x^1}{2!} + \frac{3 x^2}{3!} + \ldots \\
    &= \frac{x^0}{0!} + \frac{x^1}{1!} + \frac{x^2}{2!} + \ldots = e^x
\end{align*}

\subsection{Maclaurin series}
A Maclaurin series is a Taylor series expansion of a function about 0.

\[f\left(x\right) = f\left(0\right) + f'\left(0\right)x + f''\left(0\right)\frac{x^2}{2!} + f^{\left( 3 \right)}\left(0\right)\frac{x^3}{3!} + \ldots\]

\section{Indefinite integrals}
\subsection{Reversed chain rule}
\paragraph{Examples}
\[\int \tan x \dif x = \int \frac{\sin{x}}{\cos{x}} \dif x\]
Letting \(f(x) = \cos x\),
\[\int \frac{\sin{x}}{\cos{x}} \dif x = - \int \frac{f'(x)}{f(x)} \dif x
= - \ln{f(x)} + C = - \ln{\cos x} + C\]
Note that, in this case, \(g'(x) = x^{-1}\), and that the integral of the
reciprocal function is the natural logarithm function.

\subsection{Substitution}\label{substitution}
\paragraph{Examples}
\[\int \cos^3 x \dif x
= \int \left( 1 - \sin^2 x \right) \cos x \dif x
= \int \cos x \dif x - \int \sin^2 x \cos x \dif x\]
\[\int \cos x \dif x = \sin x + C\]
Let \(u = \sin x\), therefore \(\dif u = \cos x \dif x\).
\[\int \sin^2 x \cos x \dif x = \int u^2 \dif u = \frac{\sin^3 x}{3} + C\]
\[\int \cos x \dif x - \int \sin^2 x \cos x \dif x
= \sin x + \frac{\sin^3 x}{3} + C\]

\subsection{Trigonometric substitutions}
Some integrals require the substitution of a variable by an expression. This is
the opposite of what is done in \ref{substitution}, on which an expression is
replaced by a variable.

The following table summarizes the most effective trigonometric substitutions
and their intervals. The interval restrictions are important to ensure that the
function which defines the substitution is one-to-one \cite{calculus-stewart-2010}.

\begin{center}
    % 2 is the default, but here we want more spacing.
    \renewcommand{\arraystretch}{2}
    \begin{tabular}{ |c|c|c| }
        \hline
        Expression & Substitution & Interval \\
        \hline
        \(\sqrt{a^2 - x^2}\)
          & \(x = a\sin\theta\)
          & \(-\frac{\pi}{2} \le \theta \le \frac{\pi}{2}\) \\
        \hline
        \(\sqrt{a^2 + x^2}\)
          & \(x = a\tan\theta\)
          & \(-\frac{\pi}{2} < \theta < \frac{\pi}{2}\) \\
        \hline
        \(\sqrt{x^2 - a^2}\)
          & \(x = a\sec\theta\)
          & \(0 \le \theta < \frac{\pi}{2} \lor \pi \le \theta < \frac{3\pi}{2}\) \\
        \hline
    \end{tabular}
\end{center}

\paragraph{Example}
Find the area enclosed by an ellipse of the form
\[\frac{x^2}{a^2} + \frac{y^2}{b^2} = 1\]

Isolating \(y\) gives
\[y = \pm \frac{b}{a} \sqrt{a^2 - x^2}\]

The area of the ellipse on the first quadrant is given by
\[\int_{0}^{a} \frac{b}{a} \sqrt{a^2 - x^2} \dif x\]

This is one fourth of the total area of the ellipse, therefore
\[A = 4 \int_{0}^{a} \frac{b}{a} \sqrt{a^2 - x^2} \dif x\]

To solve the integral, let \(x = a\sin\theta\) and, consequently, \(\dif x = a
\cos \theta \dif \theta\). It is made simpler by changing the integration
limits. When \(x = 0\), \(\sin x = 0\), thus \(\theta = 0\). When \(x = a\),
\(\sin \theta = 1\), thus \(\theta = \pi / 2\).

\begin{align*}
    4 \int_{0}^{\frac{\pi}{2}} \frac{b}{a} \sqrt{a^2 - x^2} \dif x
    &= \frac{4b}{a} \int_{0}^{\frac{\pi}{2}} \sqrt{a^2 - a^2 \sin^2 \theta} a \cos \theta \dif \theta \\
    &= 4ab \int_{0}^{\frac{\pi}{2}} \sqrt{1 - \sin^2 \theta} \cos \theta \dif \theta \\
    &= 4ab \int_{0}^{\frac{\pi}{2}} \sqrt{\cos^2 \theta} \cos \theta \dif \theta \\
    &= 4ab \int_{0}^{\frac{\pi}{2}} \abs{\cos \theta} \cos \theta \dif \theta \\
    &= 4ab \int_{0}^{\frac{\pi}{2}} \cos^2 \theta \dif \theta \\
    &= 2ab \int_{0}^{\frac{\pi}{2}} 1 + \cos {2 \theta} \dif \theta \\
    &= 2ab \left[ \theta + \frac{1}{2} \sin {2 \theta} \right]_{0}^{\frac{\pi}{2}} \\
    &= 2ab \left(\frac{\pi}{2} + 0 - 0 - 0\right) \\
    &= \pi a b
\end{align*}

Which is the formula for the area of an ellipse. Furthermore, it is also the
formula for the area of a circle as a circle is an ellipse with \(a = b\).

\section{Definite integrals}
\subsection{Multiple integrals}
\paragraph{Problem.} Simplify
\[\iint_D {\left(2x - y\right) \dif A}\]
where D is bounded by the circle with center at the origin and radius 2.
\paragraph{Solution.} We rewrite the double integral as an iterated integral.
If \(x\) ranges from \(-2\) to \(2\), and \(y\), as a function of \(x\), is
bounded by \(-\sqrt{4 - x^2}\) and \(\sqrt{4 - x^2}\).
\begin{align*}
    &\int_{-2}^{2} \int_{-\sqrt{4 - x^2}}^{\sqrt{4 - x^2}} {\left(2x - y\right) \dif y} \dif x \\
    &= \int_{-2}^{2} \left[ 2xy - \frac{y^2}{2} \right]_{y = -\sqrt{4 - x^2}}^{y = \sqrt{4 - x^2}} \dif x \\
    &= \int_{-2}^{2} \left( 2x\sqrt{4 - x^2} - \frac{4 - x^2}{2} + 2x\sqrt{4 - x^2} + \frac{4 - x^2}{2} \right) \dif x \\
    &= \int_{-2}^{2} \left( 4x\sqrt{4 - x^2} \right) \dif x
\end{align*}
At this point the \(x = a\sin{t}\) trigonometric substitution is used.
\begin{align*}
    &\int_{-2}^{2} 4x\sqrt{4 - x^2} \dif x \\
    &=\int_{-\pi/2}^{\pi/2} 4 \left(a \sin{t}\right) \sqrt{a^2 - \left(a \sin{t}\right)^2} \dif t \\
    &=\int_{-\pi/2}^{\pi/2} 4 a \sin{t} \sqrt{a^2 - a^2\sin^2{t}} \dif t \\
    &=\int_{-\pi/2}^{\pi/2} 4 a^2 \sin{t} \sqrt{1 - \sin^2{t}} \dif t \\
    &=\int_{-\pi/2}^{\pi/2} 16 \sin{t} \cos{t} \dif t \\
    &=16 \left[ - \frac{\cos^2{t}}{2} \right]_{-\pi/2}^{\pi/2} \\
    &=16 \left( - \frac{\cos^2{\left(\frac{\pi}{2}\right)}}{2} + \frac{\cos^2{\left(-\frac{\pi}{2}\right)}}{2} \right) = 0
\end{align*}
