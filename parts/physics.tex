\part{Physics}

\chapter{Kinetics}
\[S = S_0 + V_0 t + \frac{a t^2}{2}\]
\[V_f^2 = V_i^2 + 2 a \Delta S\]

\chapter{Waves}
\[v = \lambda f\]

\section{Doppler effect}
\[f = f_0 \left ( \frac{v + v_r}{v + v_s} \right )\]

where

\(v\) is the velocity of waves in the medium;

\(v_r\) is the velocity of the receiver relative to the medium; positive if the
receiver is moving towards the source (and negative in the other direction);

\(v_s\) is the velocity of the source relative to the medium; positive if the
source is moving away from the receiver (and negative in the other direction).

One can remember that the velocities are positive if the receiver is chasing
the source.

\section{Sound intensity}
For a spherical sound wave, the intensity in the radial direction as a function
of distance \(r\) from the center of the sphere is given by

\[I(r) = \frac{P}{A(r)} = \frac{P}{4 \pi r^2}\]

\section{Pendulum}
The period of swing of a simple gravity pendulum depends on its length, the
local strength of gravity, and to a small extent on the maximum angle that the
pendulum swings away from vertical, called the amplitude. It is independent of
the mass of the bob. If the amplitude is limited to small swings, the period T
of a simple pendulum, the time taken for a complete cycle, is:
\[T \approx 2 \pi \sqrt{\frac{l}{g}}\]

\chapter{Optics}
\section{Mirrors and lenses}
\textbf{Focal length}
\[\reciprocal{f} = \reciprocal{p} + \reciprocal{p'}\]
A positive value for \(p'\) indicates a real image, while a negative value
indicates a virtual image.

\(f\) is positive for concave mirrors and convex lenses, and negative for convex
mirrors and concave lenses.

\textbf{Magnification}
\[m = \frac{h'}{h} = - \frac{p'}{p} = \frac{f}{f - p}\]
\(h'\) is positive if the image is upright and negative if the image is upside
down. The value of \(h\) is always positive because the object is always
upright.

\chapter{Gases}
\[P = P_A + \rho g h\]
\[\tau = P \Delta V\]

\chapter{Electricity}
\[U = R i\]
\[P = i U\]
Resistance of a parallel association of resistors:
\[\reciprocal{R_{eq}} = \sum{\reciprocal{R_i}}\]

\section{Electric force}
\[F_{el} = \frac{k Q q}{r^2}\]

\section{Transformers}
A transformer makes use of Faraday's law and the ferromagnetic properties of an
iron core to efficiently raise or lower alternating current (AC) voltages.
For an ideal transformer, the voltage ratio is equal to the turns ratio, and
power in equals power out.

\[\frac{V_1}{V_2} = \frac{N_1}{N_2}\]

\[P_{\text{in}} = P_{\text{out}}\]

\chapter{Electromagnetism}

\section{Magnetic force}
\[F_{mag} = B q v = B i l\]

\section{Magnetic fields}
\[B_{\text{wire}} = \frac{i \mu_0}{2 \pi r}\]
\[B_{\text{current loop}} = \frac{i \mu_0}{2 r}\]
\[B_{\text{solenoid}} = \frac{i \mu_0 N}{l}\]

where \(N\) is the number of coils and \(l\) is the length of the solenoid in
meters.

\section{Lenz's law}
Lenz's law is a common way of understanding how electromagnetic circuits obey
Newton's third law and the conservation of energy.
\begin{quote}
If an induced current flows, its direction is always such that it will oppose
the change which produced it.
\end{quote}
Lenz's law is shown with the negative sign in Faraday's law of induction:
\[{E}=-\frac{\partial \Phi}{\partial t}\]

\chapter{Modern physics}
\[E = hf = h\frac{c}{\lambda}\]

\section{Lorentz factor}
The Lorentz factor is the factor by which time, length, and relativistic mass
change for an object while that object is moving.
\[\gamma = \reciprocal{\sqrt{1 - \frac{v^2}{c^2}}}\]

\section{Superposition}
Superposition is a principle of quantum theory that describes a challenging
concept about the nature and behavior of matter and forces at the sub-atomic
level. The principle of superposition claims that while we do not know what the
state of any object is, it is actually in all possible states simultaneously,
as long as we don't look to check. It is the measurement itself that causes the
object to be limited to a single possibility. It can also be said that there is
no single outcome unless it is observed.

Superposition is well illustrated by \textbf{Thomas Young's double-slit
experiment}, developed in the early nineteenth century to prove that light
consisted of waves. Richard Feynman claimed that the essentials of quantum
mechanics could be grasped by an exploration of the implications of Young's
experiment.

In this experiment, a beam of light is aimed at a barrier with two vertical
slits.  The light passes through the slits and the resulting pattern is
recorded.  If one slit is covered, the pattern is what would be expected: a
single line of light, aligned with whichever slit is open. Intuitively, one
would expect that if both slits are open, the pattern of light will reflect
that fact: two lines of light, aligned with the slits.  However, what happens
is that the plate is entirely separated into multiple lines of lightness and
darkness in varying degrees.  What is being illustrated by this result is that
interference is taking place between the waves/particles going through the
slits, in what, seemingly, should be two non-crossing trajectories.

We would expect that if the beam of light particles or photons is slowed enough
to ensure that individual photons are hitting the plate, there could be no
interference and the pattern of light would be two lines of light, aligned with
the slits. In fact, however, the resulting pattern still indicates
interference, which means that, somehow, the single particles are interfering
with themselves. This seems impossible: we expect that a single photon will go
through one slit or the other, and will end up in one of two possible light
line areas. But that is not what happens. As Feynman concluded, each photon not
only goes through both slits, but simultaneously takes every possible
trajectory en route to the target.

In order to see how this might possibly occur, experiments have focused on
tracking the paths of individual photons. What happens in this case is that the
measurement in some way disrupts the photons' trajectories (in accordance with
the uncertainty principle), and somehow, the results of the experiment become
what would be predicted by classical physics: two bright lines on the
photographic plate, aligned with the slits in the barrier. Cease the attempt to
measure, however, and the pattern will again become multiple lines in varying
degrees of lightness and darkness. Each photon moves simultaneously in a
superposition of possible trajectories, and, furthermore, measurement of the
trajectory causes the superposition of states to collapse to a single position.

Text extracted from
\href{http://whatis.techtarget.com/definition/superposition}{WhatIs}.

